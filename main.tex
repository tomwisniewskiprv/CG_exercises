% ------------------------------------------------------------------------------------------------------------
% Drugi zestaw zadań z przedmiotu grafika komputerowa
% Uniwersytet Śląski - Informatyka niestacjonarna
% Tomasz Wiśniewski
% 2018-03-30
% ------------------------------------------------------------------------------------------------------------

\documentclass[a4paper]{article}

\usepackage{fullpage}
\usepackage{amsmath}
\usepackage[utf8]{inputenc}
\usepackage[T1]{fontenc}

\title{Grafika Komputerowa - II zestaw zadań}
\date{2018-03-29}
\author{Tomasz Wiśniewski}

\begin{document}
\maketitle

  \pagenumbering{gobble}
  \newpage
  \pagenumbering{arabic}
% ------------------------------------------------------------------------------------------------------------
% ZADANIE 1
% ------------------------------------------------------------------------------------------------------------
  \noindent\textbf{Zadanie 1.}
  \textnormal{Dany jest punkt o współrzędnych p(1, 2, 3). Znajdź współrzędne cylindryczne i
sferyczne tego punktu.}\\

\noindent\textnormal{Współrzędne cylindryczne:}
\begin{align*}
r &= \sqrt{x^2 + y^2} = \sqrt{1^2 + 2^2} = \sqrt{5} \approx 2.2\\
\theta &= tan^{-1}\left(\frac{y}{x}\right) , x \ne 0 , tan^{-1}\left(\frac{2}{1}\right) = 63.4349^{\circ}\\
z &= z , z = 3
\end{align*}

\noindent\textnormal{Współrzędne sferyczne:}
\begin{align*}
\rho &= \sqrt{x^2 + y^2 + z^2} = \sqrt{1^2 + 2^2 + 3^2} = \sqrt{14} \approx 3.7417\\
\phi &= sin^{-1}\left(\frac{z}{\rho}\right) = sin^{-1}\left(\frac{3}{3.7417}\right) \approx 53.19^{\circ} \\
\theta &= tan^{-1}\left(\frac{y}{x}\right) , x \ne 0 , tan^{-1}\left(\frac{2}{1}\right) = 63.4349^{\circ}
\end{align*}
% ------------------------------------------------------------------------------------------------------------
% ZADANIE 2
% ------------------------------------------------------------------------------------------------------------
\noindent\textbf{Zadanie 2.}
\textnormal{Znormalizuj wektor $V = [4, -2, 3]$}
\begin{align*}
|\vec{V}| &= \sqrt{V^2_x + V^2_y +V^2_z} = \sqrt{4^2 + (-2)^2 + 3^2} = \sqrt{29}\\
||\vec{V}|| & = \left(\frac{V_x}{|\vec{V_x}|} , \frac{V_y}{|\vec{V_y}|} , \frac{V_z}{|\vec{V_z}|}\right) = \left(\frac{4}{\sqrt{29}} , \frac{-2}{\sqrt{29}} , \frac{3}{\sqrt{29}}\right)
\end{align*}
% ------------------------------------------------------------------------------------------------------------
% ZADANIE 3
% ------------------------------------------------------------------------------------------------------------
\noindent\textbf{Zadanie 3.}
\textnormal{Znajdź kąt pomiędzy wektorami $a=[3, 2], b=[-2, 7]$}
\begin{align*}
a \cdot b &= \left|\vec{a}\right| \cdot |\vec{b}| \cdot \cos\alpha\\
\cos\alpha &= \frac{a \cdot b }{\left|\vec{a}\right| \cdot |\vec{b}|}\\
\cos\alpha &= \frac{3\cdot (-2) + 2 \cdot 7}{\sqrt[]{3^2 + 2^2} \cdot \sqrt[]{(-2)^2 +7^2}} = \frac{-6 + 14}{\sqrt[]{13} \cdot\sqrt[]{53}} = \frac{8}{\sqrt[]{689}} \approx 0.3 \\
\cos\alpha &= 72.54^{\circ}
\end{align*}
% ------------------------------------------------------------------------------------------------------------
% ZADANIE 4
% ------------------------------------------------------------------------------------------------------------
\noindent\textbf{Zadanie 4.}
\textnormal{Dane są następujące wektory: $a=[-2, 3, 2], b=[1, 1, 3]$. \\Oblicz: $ a + b$ (dodawanie wektorów) $a \cdot b$ (iloczyn skalarny) $a \times b$ (iloczyn wektorowy)}
\begin{align*}
a + b &= \left[a_x+b_x,a_y+b_y,a_z+b_z\right]=\left[ (-2) + 1, 3+1,2+3\right] = \left[-1,4,5\right]\\
a\cdot b &= a_x\cdot b_x + a_y \cdot b_y + a_z\cdot b_z = (-2) \cdot 1 + 3\cdot 1 + 2\cdot 3 = 7\\
a \times b &= \left[a_y+b_z - a_z\cdot b_y , a_z \cdot b_x - a_x \cdot b_z , a_x \cdot b_y - a_y \cdot b_x \right] = \left[ 9 - 2, 2 + 6, -2 -3\right] = \left[7,8,-5\right]
\end{align*}
% ------------------------------------------------------------------------------------------------------------
% ZADANIE 5
% ------------------------------------------------------------------------------------------------------------
\noindent\textbf{Zadanie 5.}
\textnormal{Dane są macierze M, N. Wykonaj działania: $M \cdot N , N \cdot M$.}\\
M = 
$\left[
\begin{matrix}
1 & 2 & 3\\
4 & 5 & 6
\end{matrix}
\right]$ , 
N = 
$\left[
\begin{matrix}
8 & 7 \\
6 & 5 \\
4 & 3
\end{matrix}
\right]$
% ----------------------------- MNOZENIE MACIERZY M * N
\begin{align*}
M \cdot N &=
\left[
\begin{matrix}
1 & 2 & 3 \\ 4 & 5 & 6
\end{matrix}
\right]
\cdot 
\left[
\begin{matrix}
8 & 7 \\ 6 & 5 \\ 4 & 3 
\end{matrix}
\right] = 
\left[ 
\begin{matrix}
1\cdot 8 + 2 \cdot 6 + 3 \cdot 4 & 1 \cdot 7 + 2 \cdot 5 + 3 \cdot 3 \\
4\cdot 8 + 5 \cdot 6 + 6 \cdot 4 & 4 \cdot 7 + 5 \cdot 5 + 6 \cdot 3 
\end{matrix}
\right] = 
\left[
\begin{matrix}
32 & 26 \\ 86 & 71
\end{matrix}
\right]\\
% ----------------------------- MNOZENIE MACIERZY N * M
N \cdot M &=
\left[
\begin{matrix}
8 & 7 \\ 6 & 5 \\ 4 & 3 
\end{matrix}
\right] 
\cdot
\left[
\begin{matrix}
1 & 2 & 3 \\ 4 & 5 & 6
\end{matrix}
\right]
= 
\left[ 
\begin{matrix}
8\cdot 1 + 7 \cdot 4 & 8 \cdot 2 + 7 \cdot 5 & 8 \cdot 3 + 7 \cdot 6\\
6\cdot 1 + 5 \cdot 4 & 6 \cdot 2 + 5 \cdot 5 & 6 \cdot 3 + 5 \cdot 6\\
4\cdot 1 + 3 \cdot 4 & 4 \cdot 2 + 3 \cdot 5 & 4 \cdot 3 + 3 \cdot 6
\end{matrix}
\right] = 
\left[
\begin{matrix}
32 & 51 & 66 \\ 
26 & 37 & 48 \\
16 & 23 & 30
\end{matrix}
\right]
\end{align*}

\newpage
% ------------------------------------------------------------------------------------------------------------
% ZADANIE 6
% ------------------------------------------------------------------------------------------------------------
\noindent\textbf{Zadanie 6.}
\noindent\textnormal{Dany jest wektor V oraz macierz M :} \\
$V = [-3, 1 ,2] , 
M = \left[\begin{matrix}
7 & -2 & 3 \\
4 & 1  & -5 \\
-6 & 8 & 9
\end{matrix}\right]$\\

\noindent\textnormal{Wykonaj działania : $V \cdot M, M^T \cdot V, M \cdot V^T, M^T \cdot V^T$}

\begin{align*}
V \cdot M &= \left[\begin{matrix}
-3 & 1 & 2
\end{matrix}\right]
\cdot
\left[\begin{matrix}
7 & -2 & 3 \\
4 & 1  & -5 \\
-6 & 8 & 9
\end{matrix}\right] = 
\left[\begin{matrix}
-3 \cdot 7 + 1 \cdot 4 + 2 \cdot -6 & -3 \cdot -2+ 1\cdot 1 + 2 \cdot 8 & -3 \cdot 3 + 1 \cdot -5 + 2 \cdot9
\end{matrix}\right] \\ V \cdot M &= 
\left[\begin{matrix}
-29 & 23 & 4
\end{matrix}\right] \\ 
M^T \cdot V &= niewykonalne \\
M \cdot V^T &= \left[\begin{matrix}
7 & -2 & 3 \\
4 & 1  & -5 \\
-6 & 8 & 9
\end{matrix}\right] \cdot \left[\begin{matrix}
-3 \\ 1 \\ 2
\end{matrix}\right] = 
\left[\begin{matrix}
7 \cdot -3 + -2 \cdot 1 + 3 \cdot 2 \\
4 \cdot -3 + 1 \cdot 1 + -5 \cdot 2 \\
-6 \cdot -3 + 8 \cdot 1 + 9 \cdot 2 
\end{matrix}\right] = 
 \left[\begin{matrix}
-17 \\ -21 \\ 44
\end{matrix}\right]\\
% ------------------------------------------------------------------------------------------------------------
M^T \cdot V^T &= \left[\begin{matrix}
7 & 4 & -6 \\
-2 & 1  & 8 \\
3 & -5 & 9
\end{matrix}\right] \cdot \left[\begin{matrix}
-3 \\ 1 \\ 2
\end{matrix}\right] = 
\left[\begin{matrix}
7 \cdot -3 + 4 \cdot 1 + -6 \cdot 2 \\
% * <tomasz.wisniewski.gm@gmail.com> 2018-03-30T14:33:51.642Z:
%
% ^.
-2 \cdot -3 + 1 \cdot 1 + 8 \cdot 2 \\
3 \cdot -3 + -5 \cdot 1 + 9 \cdot 2 
\end{matrix}\right] = 
 \left[\begin{matrix}
-29 \\ 23 \\ 4
\end{matrix}\right]
\end{align*}

% ------------------------------------------------------------------------------------------------------------
% ZADANIE 7
% ------------------------------------------------------------------------------------------------------------

\noindent\textbf{Zadanie 7.}
\noindent\textnormal{Jakie współrzędne będzie miał punkt p(-2,4) po translacji o wektor t=[4,8] i
obrocie o kąt $\alpha=60^{\circ}$ względem środka układu współrzędnych.} \\ \\
Przesunięcie (translacja) o wektor :
\begin{align*}
p &= (-2 , 4 , 1) , t = [ 4 , 8 , 1] \\
MP &= 
\left[
\begin{matrix}
1 & 0 & t_x \\
0 & 1 & t_y \\
0 & 0 & 1
\end{matrix}
\right] =
\left[
\begin{matrix}
1 & 0 & 4\\
0 & 1 & 8\\
0 & 0 & 1
\end{matrix}
\right]
\cdot
\left[
\begin{matrix}
-2 \\ 4 \\ 1
\end{matrix}
\right]
=
\left[
\begin{matrix}
-2 + 4 \\ 4 +8  \\ 1
\end{matrix}
\right] 
=
\left[
\begin{matrix}
2 \\ 12 \\ 1
\end{matrix}
\right] \\
\end{align*}
Obrót względem środka układu współrzędnych :
\begin{align*}
p &=
\left[
\begin{matrix}
x \\ y \\ z
\end{matrix}
\right] = 
\left[
\begin{matrix}
\left( x \cdot \cos\alpha\right) - \left( y \cdot \sin\alpha\right) \\
(x \cdot \sin\alpha) + (y \cdot \cos\alpha)\\
z
\end{matrix}
\right]
= 
\left[
\begin{matrix}
( 2 \cdot \frac{1}{2}) - ( 12 \cdot \frac{\sqrt[]{3}}{2}) \\
( 2 \cdot \frac{\sqrt[]{3}}{2}) + ( 12 \cdot \frac{1}{2})\\
z
\end{matrix}
\right]
= 
\left[
\begin{matrix}
1 - 6 \sqrt[]{3} \\
\sqrt[]{3} + 6   \\
1
\end{matrix}
\right]
= 
\left[
\begin{matrix}
-9.4 \\
7.7   \\
1
\end{matrix}
\right]
\end{align*}

% ------------------------------------------------------------------------------------------------------------
% ZADANIE 8
% ------------------------------------------------------------------------------------------------------------

\noindent\textbf{Zadanie 8.}
\noindent\textnormal{Jakie współrzędne będzie miał punkt p(3,2) po obrocie o kąt $\alpha=45^{\circ}$ względem
punktu p(2,1).}
\begin{align*}
x_1 &= (x-x_u) \cdot \cos (\alpha) - (y-y_u) \cdot \sin (\alpha)+x_u = (3 - 2) \cdot \frac{\sqrt[]{2}}{2}
- ( 2 - 1 ) \cdot \frac{\sqrt[]{2}}{2} + 2 = 2 \\
y_1 &= (x-x_u) \cdot \sin (\alpha) + (y-y_u) \cdot \cos (\alpha)+y_u = (3 - 2) \cdot \frac{\sqrt[]{2}}{2}
+ ( 2 - 1 ) \cdot \frac{\sqrt[]{2}}{2} + 1 = 1 + \sqrt[]{2} \\
p^{\prime} &= ( 2 , 1 + \sqrt[]{2}) = (2 , 2.4)
\end{align*}

% ------------------------------------------------------------------------------------------------------------
% ZADANIE 9
% ------------------------------------------------------------------------------------------------------------
\noindent\textbf{Zadanie 9.}
\noindent\textnormal{Dokonaj skalowania sześcianu o długości boku 10 przy wykorzystaniu macierzy
skalowania S. Podaj wymiary figury wyjściowej.}

\begin{align*}
S &= 
\left[
\begin{matrix}
1 & 0 & 0 & 0\\
0 & 2 & 0 & 0\\
0 & 0 & 5 & 0\\
0 & 0 & 0 & 1
\end{matrix}
\right]
\end{align*}

\noindent Współrzedne przodu i tyłu:
\begin{align*}
p &= 
\left[
\begin{matrix}
0 & 0 & 0 
\end{matrix}
\right] ,
t = 
\left[
\begin{matrix}
10 & 10 & 10 
\end{matrix}
\right]
\end{align*}
Skalowanie :
\begin{align}
p &= 
\left[
\begin{matrix}
0 \\ 0 \\ 0 \\ 1
\end{matrix}
\right] \cdot 
\left[
\begin{matrix}
1 & 0 & 0 & 0\\
0 & 2 & 0 & 0\\
0 & 0 & 5 & 0\\
0 & 0 & 0 & 1
\end{matrix}
\right] =
\left[
\begin{matrix}
0 \\ 0 \\ 0 \\ 1
\end{matrix}
\right] \\
t &= 
\left[
\begin{matrix}
10 \\ 10 \\ 10 \\ 1
\end{matrix}
\right] \cdot 
\left[
\begin{matrix}
1 & 0 & 0 & 0\\
0 & 2 & 0 & 0\\
0 & 0 & 5 & 0\\
0 & 0 & 0 & 1
\end{matrix}
\right] =
\left[
\begin{matrix}
10 \\ 20 \\ 50 \\ 1
\end{matrix}
\right] 
\end{align}

\noindent Wymiary wyjściowe:
\begin{align}
a = 10 \\
b = 20 \\
h = 50 
\end{align}


% ------------------------------------------------------------------------------------------------------------
% ZADANIE 10
% ------------------------------------------------------------------------------------------------------------
\noindent\textbf{Zadanie 10.}
\noindent\textnormal{Znajdź wektor normalny do następującego wielokąta:}\\
$P_1 = (4,8,2) = (x_1,y_1,z_1)\\
P_2 = (2 ,5 , 2) = (x_2,y_2,z_2) \\
P_3 = (5 , 2 , 1)= (x_3,y_3,z_3) $

\begin{align*}
\vec{u} &= \left[u_1 , u_2 , u_3\right] = \left[x_1 - x_2 , y_1 - y_2 , z_1-z_2\right] = \left[2 , 3 , 0\right] \\
\vec{v} &= \left[v_1 , v_2 , v_3\right] = \left[x_3 - x_2 , y_3 - y_2 , z_3-z_2\right] = \left[3 , -3 , -1\right]\\
\vec{n} &= \left[u_2v_3 - u_3v_2 , u_3v_1 - u_1v_3 , u_1v_2 - u_2v_1\right] = \left[-3 , 2 , -15 \right]
\end{align*}

% ------------------------------------------------------------------------------------------------------------
% ZADANIE 11
% ------------------------------------------------------------------------------------------------------------
\noindent\textbf{Zadanie 11.}
\noindent\textnormal{Siatka geometryczna obiektu 3D zawiera punkty $p_1(0,0,0)$, $p_2(5,-5,5)$, $p_3(3,4,5)$,
$p4(-4,-2,-1)$. Oblicz pozycję punktów $p'_1$, $p'_2$, $p'_3$, $p'_4$, które powstaną po zastosowaniu do nich transformacji skalowania macierzą M. Punktem centralnym, według którego przeprowadzone jest skalowanie, jest punkt o współrzędnych p(3,4,1).}\\

\noindent Macierz skalowania M oraz macierz jednokładności J:
\begin{align*}
M &= 
\left[
\begin{matrix}
3 & 0 & 0 & 0\\
0 & 1 & 0 & 0\\
0 & 0 & 2 & 0\\
0 & 0 & 0 & 1
\end{matrix}
\right] , 
J = 
\left[
\begin{matrix}
k_1 & 0 & 0 & 0\\
0 & k_2 & 0 & 0\\
0 & 0 & k_3 & 0\\
(1-k_1) \cdot x & (1-k_2) \cdot y & (1-k_3) \cdot z & 1
\end{matrix}
\right]
=
\left[
\begin{matrix}
3 & 0 & 0 & 0\\
0 & 1 & 0 & 0\\
0 & 0 & 2 & 0\\
-6 & 0 & -1 & 1
\end{matrix}
\right]
\end{align*}
% ------------------------------------------------------------------------------------------------------------
Skalowanie (mnożenie punktu przez macierz J):\\
\begin{align*}
% ------------------------------------------------------------------------------------------------------------
p_1 &= 
\left[
\begin{matrix}
0 & 0 & 0 & 1
\end{matrix}
\right] \cdot
\left[
\begin{matrix}
3 & 0 & 0 & 0\\
0 & 1 & 0 & 0\\
0 & 0 & 2 & 0\\
-6 & 0 & -1 & 1
\end{matrix}
\right] = 
\left[
\begin{matrix}
-6 & 0 & -1 & 1
\end{matrix}
\right] 
=
\left(
-6 , 0 , -1
\right) \\
% ------------------------------------------------------------------------------------------------------------
p_2 &= 
\left[
\begin{matrix}
5 & -5 & 5 & 1
\end{matrix}
\right] \cdot
\left[
\begin{matrix}
3 & 0 & 0 & 0\\
0 & 1 & 0 & 0\\
0 & 0 & 2 & 0\\
-6 & 0 & -1 & 1
\end{matrix}
\right] = 
\left[
\begin{matrix}
9 & -5 & 9 & 1
\end{matrix}
\right] 
=
\left(
9, -5 , 9
\right) \\
% ------------------------------------------------------------------------------------------------------------
p_3 &= 
\left[
\begin{matrix}
3 & 4 & 5 & 1
\end{matrix}
\right] \cdot
\left[
\begin{matrix}
3 & 0 & 0 & 0\\
0 & 1 & 0 & 0\\
0 & 0 & 2 & 0\\
-6 & 0 & -1 & 1
\end{matrix}
\right] = 
\left[
\begin{matrix}
3 & 4 & 9 & 1
\end{matrix}
\right] 
=
\left(
3 , 4 , 9
\right) \\
% ------------------------------------------------------------------------------------------------------------
p_4 &= 
\left[
\begin{matrix}
-4 & -2 & -1 & 1
\end{matrix}
\right] \cdot
\left[
\begin{matrix}
3 & 0 & 0 & 0\\
0 & 1 & 0 & 0\\
0 & 0 & 2 & 0\\
-6 & 0 & -1 & 1
\end{matrix}
\right] 
= 
\left[
\begin{matrix}
-18 & -2 & -3 & 1
\end{matrix}
\right] 
=
\left(
-18 , -2 ,-3
\right) 
\end{align*}


% ------------------------------------------------------------------------------------------------------------
% ZADANIE 12
% ------------------------------------------------------------------------------------------------------------
\noindent\textbf{Zadanie 12.}
\noindent\textnormal{W trójwymiarowej przestrzeni euklidesowej znajduje się obiekt, którego środek
znajduje się w punkcie $p(107.183, 934.013, 12.781)$. Obiekt ten rozpoczyna
swój ruch według wektora prędkości $\vec{v}=[1, -2, -2]$ i kontynuuje go przez okres
czasu dt=1. Po tym okresie wektor prędkości wykonuje operację rotacji o kąt
$\alpha=90^{\circ}$ względem osi Z. Po wykonaniu obrotu zwiększana jest długość wektora $\vec{v}$
o wartość 1.5 i kontynuowany jest ruch obiektu przez okres czasu dt=2. Wyznacz końcową pozycję punktu $p'$ będącą środkiem poruszanego obiektu.}\\

\noindent$p = (107.183, 934.013, 12.781)$\\
$\vec{v}=[1, -2, -2]$	\\
$\alpha=90^{\circ}$		\\ \\
Przesunięcie punktu p wektor $\vec{v}$ w czasie t:
\begin{align*}
\Delta t &= 1\\
x(t) &= x + V_x\cdot t = 107.183 + (1\cdot 1) = 108.183\\
y(t) &= y + V_y\cdot t = 934.013 + (-2 \cdot 1) = 932.013\\
z(t) &= z + V_z\cdot t = 12.781 + (-2 \cdot 1) = 10.781
\end{align*}
Rotacja o kąt $\alpha= 90^{\circ}$:
\begin{align*}
R_z(\alpha) &= 
\left[
\begin{matrix}
\cos\alpha & \sin\alpha & 0 & 0\\
-\sin\alpha & \cos\alpha & 0 & 0\\
0 & 0 & 1 & 0\\
0 & 0 & 0 & 1
\end{matrix}
\right] 
=
\left[
\begin{matrix}
0 & 1 & 0 & 0\\
-1 & 0 & 0 & 0\\
0 & 0 & 1 & 0\\
0 & 0 & 0 & 1
\end{matrix}
\right] 
\\
p^{\prime} &= \left[\begin{matrix}
 108.183 & 932.013 & 10.781 & 1
\end{matrix}\right] \cdot
\left[
\begin{matrix}
0 & 1 & 0 & 0\\
-1 & 0 & 0 & 0\\
0 & 0 & 1 & 0\\
0 & 0 & 0 & 1
\end{matrix}
\right] = 
\left[\begin{matrix}
% ------------------------------------------------------------------------------------------------------------
 -932.013 & 108.183 & 10.781
 % ------------------------------------------------------------------------------------------------------------
\end{matrix}\right]
\end{align*}
Zwiększenie długości wektora $\vec{v}$ (skalowanie):
\begin{align*}
\vec{v} &= \left[\begin{matrix}
 1 & -2 & -2
\end{matrix}\right] \\
k &= 1.5 \\
\vec{v} &= \left[\begin{matrix}
 k\cdot1 & k\cdot(-2) & k\cdot (-2)
\end{matrix}\right]
=
\left[\begin{matrix}
1.5 & -3 & -3
\end{matrix}\right]\\
\end{align*}
Kontynuacja ruchu obiektu w czasie $\Delta t = 2 $, ponowna translacja o wektor $\vec{v}$:\\
\begin{align*}
\Delta t &= 2\\
x(t) &= x + V_x\cdot t = -932.013 + (1.5\cdot 2) = -929.013\\
y(t) &= y + V_y\cdot t = 108.183 + (-3 \cdot 2) = 102.183\\
z(t) &= z + V_z\cdot t = 10.781 + (-3 \cdot 2) = 4.781
\end{align*}


\end{document}
